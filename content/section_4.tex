\chapter{Conclusion}
\label{chap:conclusion}

In conclusion, this report compared two approaches to developing multi-client real-time synchronized web applications: TEASync and React, NodeJs, Express, and other libraries. The comparison covered various aspects, including developer experience, implementation, and user interface. TEASync offers a unique approach to building real-time collaborative applications, simplifying state management and data synchronization. However, it lacks native HTML API support and requires more effort to create professional-looking UI components. On the other hand, React, NodeJs, Express, and other libraries provide a more traditional and widely adopted approach, offering robust and customizable UI components, but with a steeper learning curve and more complex setup.

The choice between these approaches depends on the project requirements and the development team’s expertise. TEASync is suitable for rapid prototyping and small-scale applications, while React and its ecosystem are better suited for large-scale and complex applications. This comparison provides valuable insights for developers and researchers exploring different approaches to building real-time collaborative web applications. Future work includes further evaluating the performance and scalability of TEASync and exploring its potential applications in various domains.