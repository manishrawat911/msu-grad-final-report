\chapter{React Components, Redux State Management, Express Server}
% {React Components, Redux State Management, Express Server}
\label{appendix:B}

\section{Chatroom}

\begin{lstlisting}[language=Java, caption={Login Page}, label={lst:java}]
'use client'

import { useRouter } from "next/navigation";
import { useAppDispatch } from "../hooks";
import { setUserInfo } from "../global.state";

export default function LoginPage() {
    const router = useRouter();
    const dispatch = useAppDispatch();

    const handleSubmit = async (formData: any) => {
        const username = formData.get("username");
        const password = formData.get("password");
        const ob_json = {
            username: username,
            password: password
        }
        try {
            const response = await fetch('http://localhost:5050/login/admin', {
                method: 'POST',
                headers: {
                    'Content-Type': 'application/json'
                },
                body: JSON.stringify(ob_json)
            });

            if (!response.ok) {
                throw new Error('Network response was not ok');
            }

            dispatch(setUserInfo({username:  username, userType: "admin", isLoggedIn: true}));
            const data = await response.json();
            console.log(data); // Handle the response data
            router.push('/session/new') 
        } catch (error) {
            console.error('Error:', error);
        }
    }

    return (
        <div className="flex min-w-full h-[90%] bg-gray-100 justify-center items-center content-center">
            <div className="container 
                            flex 
                            flex-col 
                            rounded-lg
                            p-5 
                            w-fit h-fit 
                            bg-slate-600
                            justify-center 
                            items-center">
                <form className="flex flex-col space-y-0.5 content-center items-center" action={handleSubmit}>
                    <h1 className="text-center text-slate-100 ">Login</h1>
                    <input className="rounded p-2" type='text' id='username' placeholder="username" name='username' required/><br />
                    <input className="rounded p-2" type='password' id='password' placeholder="password" name='password' required/><br />
                    <button className="p-1 w-1/2 text-green-950 font-bold bg-green-400 rounded place-self-center" type="submit">Submit</button>
                </form>
            </div>
            <p></p>
        </div>
    )
}
\end{lstlisting}
\begin{lstlisting}[language=Java, caption={New Session Page}, label={lst:java}]
'use client'

import { selectIsLoggedIn, selectSessionID, selectUserType } from "/app/global.state";
import { useAppDispatch, useAppSelector } from "/app/hooks";
import { useRouter } from "next/navigation";
import { useEffect, useState } from "react";
import { setSessionId } from "/app/global.state";

let nextId = 1;

export default function NewSession(){

    const dispatch = useAppDispatch();
    const loggedIn = useAppSelector(selectIsLoggedIn);
    const userType = useAppSelector(selectUserType);
    const sessionId = useAppSelector(selectSessionID);
    
    const router = useRouter();

    const [participants, setParticipants] = useState([]);
    const addParticipants = (formData: any) =>{
        let username = formData.get('username');
        let password = formData.get('password');
        
        for(let i=0;i<participants.length;i++){
            console.log(participants[i].username);

            if(participants[i].username === username){
                alert(`User ${username} already exists!`);
                return;
            }
        }
        setParticipants([...participants, {username: username , password : password }]);
    };

    const startSession = async () =>{
        console.log("trying to start session");
        const ob_json = {
            username: "admin",
            participants: participants
        }
        console.log(`session data: ${JSON.stringify(ob_json)}`);
        try {
            const response = await fetch('http://localhost:5050/session', {
                method: 'POST',
                headers: {
                    'Content-Type': 'application/json'
                },
                body: JSON.stringify(ob_json)
            });

            if (!response.ok) {
                throw new Error('Network response was not ok');
            }

            const data = await response.json();
            dispatch(setSessionId(data.sessionId));
            console.log(data); // Handle the response data
            router.push(`/chatroom/${data.sessionId}`) 
        } catch (error) {
            console.error('Error:', error);
        }
    };
    return (
        <div className="container h-5/6 flex flex-col justify-center items-center bg-slate-300">
            <h1>Add  Participants</h1>
            <div className="flex space-x-1">
                <form className="flex flex-col space-y-1" action={addParticipants}>
                    <input required name="username" type="text" placeholder="username"></input>
                    <input required name="password" type="password" placeholder="password"></input>
                    <button className="bg-blue-500 text-white rounded" type="submit">Add</button>
                </form>
                <button className="bg-green-500 p-3 text-white rounded" onClick={startSession}>Go!</button>
            </div>
            <p>List of participants</p>
            <ul>
                {participants.map( participant =>
            <li>username: {participant.username}, password: {participant.password}</li>
            )}
            </ul>
        </div>
    )
}
\end{lstlisting}
\begin{lstlisting}[language=Java, caption={Message Input Component}, label={lst:java}]
export default function MessageInput({send}) {
    return (
        <div className="h-1/6 bg-blue-400 flex p-2 space-x-1">
            <form className="flex basis-full space-x-1" action={send}>
                <input name="message" className="basis-5/6 placeholder:italic" placeholder="type..." type="text" />
                <button type="submit" className="basis-1/6 bg-green-800 w-1/6 h-full rounded text-fuchsia-100">Send</button>
            </form>
        </div>
    )
}
\end{lstlisting}
\begin{lstlisting}[language=Java, caption={Message View Component}, label={lst:java}]
'use client'
export default function MessagesView({ messages }) {
    return (
        <>
            <div className="flex flex-col h-4/6 bg-red-500">
                {JSON.stringify(messages)}
            </div>
        </>
    )
}
\end{lstlisting}
\begin{lstlisting}[language=Java, caption={Chat Page Component}, label={lst:java}]
'use client'
import { SocketProvider, useSocket } from "/app/SocketProvider";
import { selectIsLoggedIn, selectSessionID, selectUserType, selectUsername, setUserInfo } from "/app/global.state";
import { useAppDispatch, useAppSelector } from "/app/hooks";
import MessageInput from "/lib/components/chat/message_input";
import MessagesView from "/lib/components/chat/messages_view";
import { time } from "console";
import { stringify } from "querystring";
import { useEffect, useState } from "react";

export default function ChatPage({ params }: { params: { id: string } }) {
    const dispatch = useAppDispatch();
    const loggedIn = useAppSelector(selectIsLoggedIn);
    const userType = useAppSelector(selectUserType);
    const sessionId = useAppSelector(selectSessionID);

    if (!sessionId) {
        dispatch(setUserInfo({ sessionId: params.id }));
    }
    if (!loggedIn && userType != "admin") return (
        <ParticipantLogin />
    )
    return (
        <>
            <div className="container mx-auto 2xl:max-w-3xl justify-center items-center">
                <h1>Chat room id:  {params.id}</h1>
                <SocketProvider>
                    <ChatView userType={userType} />
                </SocketProvider>
            </div>
        </>
    )
}

function ChatView({ userType }: { userType: string }) {
    const [chatHistory, setChatHistory] = useState([]);
    const username = useAppSelector(selectUsername)
    const { socket, isConnected } = useSocket();
    const sessionId = useAppSelector(selectSessionID)
    const date = new Date();

    function sendMessage(formData) {
        const json_ob = {
            sender: username,
            roomId: sessionId,
            message: formData.get("message")
        }
        if(socket)
            socket.emit('sendMessage', json_ob);
    }
    useEffect(() => {
        if (socket) {
            if (userType == 'user') {
                socket.emit('joinRoom', { sessionId, username });
                console.log("joining session to chat");
            }
            else if (userType == 'admin') {
                socket.emit('createRoom', { sessionId, username })
                console.log(`session creation initiated by admin `)
            }
            socket.on('roomCreated', (data) => {
                console.log(`Room created: ${JSON.stringify(data)}`);
            });
    
            socket.on('joinedRoom', (data) => {
                console.log('Joined room');
                setChatHistory((previousState) => {
                console.log(`previous messages: ${JSON.stringify(data)}`)
                   return data.chatHistory
                });
            });
            socket.on('newMessage', (data) => {
                console.log(`New message : ${JSON.stringify(data)}`);
                setChatHistory((previousChatHistory) => {
                    console.log(`Previous messages so far: ${JSON.stringify(previousChatHistory)}`);
                    return [...previousChatHistory, data];
                })
            });
            socket.on('userJoined', (data) => {
                console.log(`New user joined: ${JSON.stringify(data)}`)
            });
            socket.on('userLeft', (data) => {
                console.log(`User left:, ${stringify(data)}`);
            });
        }
        if (!isConnected) {
            console.log('Unable to connect to the server')
        }

    }, [socket]);
    const messageHolder = chatHistory.map(message => {
        return (
            <div className="flex flex-col-reverse bg-slate-100 rounded m-1 p-2">
                <p className="text-blue-800 text-sm/[9px]">{message.username}</p>
                <p className="font-bold">{message.message}</p>
            </div>
        )
    });

    return (
        <>
            <div className="container flex flex-col h-4/6 bg-stone-400 rounded p-2 justify-center">
                {messageHolder}
            </div>
            <MessageInput send={sendMessage}/>
        </>
    )
}

function ParticipantLogin() {
    const dispatch = useAppDispatch();
    const sessionId = useAppSelector(selectSessionID);

    const handleSubmit = async (formData: any) => {
        const username = formData.get("username");
        const password = formData.get("password");
        const ob_json = {
            username: username,
            password: password,
            sessionId: sessionId
        }
        try {
            const response = await fetch('http://localhost:5050/login/participant', {
                method: 'POST',
                headers: {
                    'Content-Type': 'application/json'
                },
                body: JSON.stringify(ob_json)
            });
            if (!response.ok) {
                console.log(response.body)
                throw new Error('Network response was not ok');
            }
            const data = await response.json();
            dispatch(setUserInfo({ username: username, userType: "user", isLoggedIn: true, sessionId: sessionId }));
        } catch (error) {
            console.error('Error:', error);
        }
    }
    return (
        <div className="flex min-w-full h-full bg-gray-100 justify-center items-center content-center">
            <div className="container 
                            flex 
                            flex-col 
                            rounded-lg
                            p-5 
                            w-fit h-fit 
                            bg-slate-600
                            justify-center 
                            items-center">
                <form className="flex flex-col space-y-0.5 content-center items-center" action={handleSubmit}>
                    <h1 className="text-center text-slate-100 ">Participant Login</h1>
                    <input className="rounded p-2" type='text' id='username' placeholder="username" name='username' required /><br />
                    <input className="rounded p-2" type='password' id='password' placeholder="password" name='password' required /><br />
                    <button className="p-1 w-1/2 text-green-950 font-bold bg-green-400 rounded place-self-center" type="submit">Submit</button>
                </form>
                <p className="text-sm text-gray-300">Enter password provided by instructor or TA</p>
            </div>
        </div>
    )
}
\end{lstlisting}
\begin{lstlisting}[language=Java, caption={Socket Provider Component}, label={lst:java}]
'use client'
import { createContext, useContext, useEffect, useState } from "react";
import { io as ClientIO } from 'socket.io-client'
import { useAppSelector } from "./hooks";
import { selectSessionID } from "./global.state";

export type SocketContextType = {
    socket: any | null,
    isConnected: boolean
}

const defaultSocket = {
    socket: null,
    isConnected: false
};

export const SocketContext = createContext<SocketContextType>(defaultSocket);

export const useSocket = () => {
    return useContext(SocketContext);
}


export const SocketProvider= ({ children }: { children: React.ReactNode }) => {

    const [socket, setSocket] = useState<any | null>(null);
    const [isConnected, setIsConnected] = useState(false);

    useEffect(() => {

        const socketInstance = new (ClientIO as any)(`http://localhost:5050`); // 
        // Add an event listener to handle connections
        socketInstance.on('connect', () => {  // explain what happening here
            console.log("Websocket connected    !");
            setIsConnected(true);
            
        });
        socketInstance.on('disconnect', (reason: any) => {
            console.warn(`Websocket disconnected because of ${reason}`);
            setIsConnected(false);
        });
        setSocket(socketInstance);
        //Clean up function that gets called whenever this component unmounts
        return () => {
            console.log("Disconnecting websocket...");
            // Close the connection when this component unmounts
            socketInstance.disconnect();
        };
     }, []);
    return (
        <SocketContext.Provider value={{socket: socket, isConnected: isConnected}}>
            {children}
        </SocketContext.Provider>
    )
}
\end{lstlisting}
\begin{lstlisting}[language=Java, caption={Application State}, label={lst:java}]
import { createSlice } from "reduxjs/toolkit"

export interface CustomizedAppConfig {
    isLoggedIn: boolean,
    username: string,
    userType: "admin" | "user" | "unknown"
    sessionId: string
}

const defaultConfig: CustomizedAppConfig = {
    isLoggedIn: false,
    username: "",
    userType: "unknown",
    sessionId: ""
}

export const globalSlice = createSlice(
    {
        name: "global",
        initialState: defaultConfig,
        reducers: {
            setUserInfo: (state, action) => {
                return { ...state, ...action.payload };
            },
            setSessionId: (state, action) => {
                return { ...state, sessionId: action.payload}
            },
            logout: state => { 
                return defaultConfig;    
            }
        }
    })

// Action creators are generated for each case reducer function
export const { setUserInfo, setSessionId, logout } = globalSlice.actions

export const selectIsLoggedIn = (state: any): boolean => state.global.isLoggedIn
export const selectUsername = (state: any): string => state.global.username
export const selectUserType = (state: any): "admin"|"user" => state.global.userType
export const selectSessionID = (state:  any):string=> state.global.sessionId

export default globalSlice.reducer      
\end{lstlisting}
\begin{lstlisting}[language=Java, caption={Express Backend Server}, label={lst:java}]
import { Session } from "inspector";

const express = require('express');
const http = require('http');
const socketIo = require('socket.io');
const cors = require('cors');
const bodyParser = require('body-parser');


const app = express();
app.use(cors());
const server = http.createServer(app);
const port = 5050;
const io = socketIo(server, {
    cors: {
        origin: "*", // Update this with your client's domain
        methods: ["GET", "POST"]
    }
});

app.use(bodyParser.json());

const users: L_User[] = [
    { username: "admin", password: "1234" }
]

const userSessions: Map<string, string> = new Map();
const sessionParticipants: L_Session[] = [];
const sessionMessages: Map<string, string[]> = new Map();
const activeRooms = {};


app.post("/login/admin", (request, response) => {
    const { username, password } = request.body;
    const user = users.find(user => user.username === username && user.password === password);

    if (user) {
        response.json({ message: 'Login successful' });
    } else {
        response.status(401).json({ error: 'Invalid username or password' });
    }
})

app.post("/login/participant", (request, response) => {
    const { username, password, sessionId } = request.body;
    const r_session = sessionParticipants.find((session) => session.id == sessionId)

    if(!r_session){
        response.status(401).json({ error: 'Invalid session' });
    }

    const user = r_session?.participants.find(user => user.username === username && user.password === password);

    if (user) {
        response.json({ message: 'Login successful' });
    } else {
        response.status(401).json({ error: 'Invalid username or password' });
    }
})

app.get("/session", (request, response) => {
    const { username } = request.body;
    const user = users.find(user => user.username === username);
    
    if (user) {
        const session = userSessions.get(user.username);
        if (session) {
            response.json({ session });
        } else {
            response.status(404).json({ error: 'Session not found' });
        }
    }
    else {
        response.status(404).json({ error: 'User not found' });
    }
});

app.post("/session", (request, response) => {
    const data = request.body;
    console.log(data);
    const user = users.find(user => user.username === data.username);

    if (user) {
        const sessionId = Math.random().toString(36).substring(7);
        const participants = data.participants;

        if (participants) {
            const temp_users: L_User[] = [];
            for (const p of participants) {
                temp_users.push({ username: p.username, password: p.password })
            }
            const session: L_Session = { id: sessionId, participants: temp_users };
            console.log(`Session create: ${sessionId}, Users: ${temp_users.toString()}`)
            sessionParticipants.push(session);
            userSessions.set(user.username, sessionId);
        }
        response.json({ sessionId });
    } else {
        response.status(404).json({ error: 'User not found' });
    }
});

app.get('/chatroom/:sessionId', (req, res) => {
    const sessionId = req.params.sessionId;
    if (!sessionId) {
      return res.status(400).send('Missing sessionId parameter');
    }
  
    // Create a new namespace for the chatroom with the sessionId
    const chatroom = io.of(`/chatroom/${sessionId}`);
  
    chatroom.on('connection', (socket) => {
      console.log(`Socket connected to chatroom: ${sessionId}`);
  
      // Handle socket events specific to this chatroom
      socket.on('message', (data) => {
        console.log(`Message received in ${sessionId}: ${data}`);
        // Broadcast the message to all connected sockets in the chatroom
        chatroom.emit('message', data);
      });
  
      // Handle socket disconnection
      socket.on('disconnect', () => {
        console.log(`Socket disconnected from chatroom: ${sessionId}`);
      });
    });
  
    // Send a success response to the UI
    res.send('Chatroom created');
  });

io.on('connection', (socket) => {
    let roomId;

    socket.on('createRoom', (payload) => {

        activeRooms[payload.sessionId] = { users: [payload.username], messages: [] };
        socket.join(payload.sessionId);
        socket.emit('roomCreated', payload);
        console.log(`Payload recieved:: ${JSON.stringify(payload)}`);
    });

    socket.on('joinRoom', (data) => {

        const room = activeRooms[data.sessionId];
        if (room) {
            roomId = data.sessionId;
            socket.join(roomId);
            room.users.push(data.username);
            
            // Send chat history to the new user
            const chatHistory = room.messages.map(({ username, message }) => ({ username: username, message }));
            socket.emit('joinedRoom', { chatHistory });
            
            io.to(roomId).emit('userJoined', { userId: socket.id });
        }
    });

    // Handler for sending a message
    socket.on('sendMessage', (data) => {
        const { sender, roomId, message } = data;
        const room = activeRooms[roomId];
        if (room) {
            room.messages.push({ username: sender, message });
            io.to(roomId).emit('newMessage', { username: sender, message });
        }
    });

    // Handler for disconnecting
    socket.on('disconnect', () => {
        if (roomId) {
            const room = activeRooms[roomId];
            if (room) {
                room.users = room.users.filter(userId => userId !== socket.id);
                io.to(roomId).emit('userLeft', { userId: socket.id });
                if (room.users.length === 0) {
                    console.log(`deleting room: ${roomId} since total  users are: ${room.users.length}`)
                    delete activeRooms[roomId];
                }
            }
        }
    });
});

server.listen(port, () => {
    console.log(`Server is running on port ${port}`);
});

type L_User = {
    username: string,
    password: string,
}

type L_Session = {
    id: string
    participants: L_User[]
}

\end{lstlisting}


\section{Calendar-Sync}
\vspace{6pt}
\begin{lstlisting}[language=Java, caption={Calendar Page}, label={lst:java}]
'use client'
import { Calendar, momentLocalizer, Event, Views } from 'react-big-calendar';
import withDragAndDrop from 'react-big-calendar/lib/addons/dragAndDrop';
import moment from 'moment';
import { useCallback, useEffect, useRef, useState } from 'react';
import { v4 as uuidv4 } from 'uuid';

import {
    Dialog,
    DialogContent,
    DialogFooter,
    DialogHeader,
    DialogTitle,
} from "/components/ui/dialog"
import { Button } from '/components/ui/button';
import { Label } from '/components/ui/label';
import { Input } from '/components/ui/input';
import { useSocket } from '/app/SocketProvider';

const localizer = momentLocalizer(moment)
const DnDCalendar = withDragAndDrop(Calendar);

interface MyEvent extends Event {
    title: string;
    resourceId: string;
}


export default function Page({ params }: { params: { id: string } }) {
    const [events, setEvents] = useState<MyEvent[]>();
    const [newEvent, setNewEvent] = useState<MyEvent | null>(null);
    const [open, setOpen] = useState(false);
    const [date, setDate] = useState(new Date(2015, 3, 1));
    const [view, setView] = useState(Views.WEEK);

    const { socket, isConnected } = useSocket();
    const socketRef = useRef(socket);

    const onNavigate = useCallback((newDate) => setDate(newDate), [setDate])
    const onView = useCallback((newView) => setView(newView), [setView])

    useEffect(() => {
        if (socket) {
            socket.emit('joinRoom', params.id);
            socket.on('updateEvents', (newEvents) => {
                console.log(`New update received: ${JSON.stringify(newEvents)}`);
                setEvents(newEvents.map((event) => ({
                    ...event,
                    start: new Date(event.start),
                    end: new Date(event.end)
                })));
            });
            socket.on('allEvents', (allEvents: MyEvent[]) => {
                console.log(`All events received: ${JSON.stringify(allEvents)}`);
                setEvents(allEvents.map((event) => ({
                    ...event,
                    start: new Date(event.start),
                    end: new Date(event.end)
                })));
                setEvents([...allEvents]);
            })
        }
        socketRef.current = socket;
    }, [socket])

    function saveEvent() {
        console.log(`Saving new event: ${newEvent.title}, ${newEvent.start}, ${newEvent.end} `);
        setEvents((oldEvents) => {
            if (socketRef.current) {
                socket.emit('updateEvents', {
                    sessionId: params.id,
                    newEvents: [...oldEvents, newEvent]
                });
                console.log(`Socket available!`)
            }
            return [...oldEvents, newEvent];
        });
        setOpen(false);
    }


    function onSelectSlot(event) {
        if (event.action === 'doubleClick') {
            console.log(`Slot selected ${{ ...event }}`);
            setNewEvent({
                //generate random id for the event
                resourceId: uuidv4(),
                title: 'New Event',
                start: event.start,
                end: event.end,
            });
            if (socketRef.current) {
                console.log(`Socket available!`)

            }
            setOpen(true);
        }
    }

    function resizeEvent(event) {
        let updatedEvent = events.find((e) => e.resourceId === event.event.resourceId);
        if (updatedEvent) {
            updatedEvent!.start = event.start;
            updatedEvent!.end = event.end;
            setEvents((oldEvents) => {
                return oldEvents.map((e) => {
                    if (e.resourceId === event.event.resourceId) {
                        return updatedEvent;
                    }
                    return e;
                });
            }); params
        }
        if (socketRef.current) {
            socket.emit('updateEvents', {
                sessionId: params.id,
                newEvents: events
            });
            console.log(`Socket available!`)
        }
        console.log(`Resized event title: ${event.event.title}`);
        console.log(`Resized event resourceId: ${event.event.resourceId}`);
        console.log(`Resized event start: ${event.start}`);
        console.log(`Resized event end: ${event.end}`);
    }
    return (
        <>
            <h1>{params.id}</h1>
            <Dialog open={open} onOpenChange={setOpen}>
                <Button variant="outline">Edit Profile</Button>
                <DialogContent className="sm:max-w-[425px]">
                    <DialogHeader>
                        <DialogTitle>Add Event</DialogTitle>
                    </DialogHeader>
                    <div className="grid gap-4 py-4">
                        <div className="grid grid-cols-4 items-center gap-4">
                            <Label htmlFor="name" className="text-right">
                                Title
                            </Label>
                            <Input
                                value={newEvent?.title}
                                onChange={(e) => setNewEvent({ ...newEvent, title: e.target.value })}
                                className="col-span-3"
                            />
                        </div>
                        <div className="grid grid-cols-4 items-center gap-4">
                            <Label htmlFor="start" className="text-right">
                                Starts at
                            </Label>
                            <Input value={newEvent?.start} readOnly
                                id="start"
                                defaultValue="peduarte"
                                className="col-span-3"
                            />
                            <Label htmlFor="end" className="text-right">
                                Ends at
                            </Label>
                            <Input value={newEvent?.end} readOnly
                                id="end"
                                defaultValue="peduarte"
                                className="col-span-3"
                            />
                        </div>
                    </div>
                    <DialogFooter>
                        <Button type="submit" onClick={saveEvent}>Save changes</Button>
                    </DialogFooter>
                </DialogContent>
            </Dialog>
            <DnDCalendar
                localizer={localizer}
                events={events}
                selectable='true'
                onSelectSlot={onSelectSlot}
                defaultView='week'
                views={['week']}
                draggableAccessor={(event) => true}
                startAccessor="start"
                endAccessor="end"
                onView={onView}
                onNavigate={onNavigate}
                resizable={true}
                onEventResize={resizeEvent}
                onEventDrop={resizeEvent}
            />
        </>
    )
}
\end{lstlisting}
\begin{lstlisting}[language=Java, caption={Welcome Page}, label={lst:java}]
'use client'

import { Button } from '/components/ui/button';
import {
    Card,
    CardContent,
    CardDescription,
    CardFooter,
    CardHeader,
    CardTitle
} from '/components/ui/card';
import { Input } from '/components/ui/input';
import { Label } from '/components/ui/label';
import { Merge, SquarePlus } from 'lucide-react';
import { useRouter } from 'next/navigation';
import React from 'react';

export default function WelcomePage() {
    const router = useRouter();
    const handleCreateSession = async () => {

        try{
            const response = await fetch('http://localhost:5050/session', {method:'POST'});
            if (!response.ok) {
                throw new Error('Network response was not ok');
            }
            const data = await response.json();
            router.push(`/session/${data.sessionId}`);
        }
        catch (error) {
            console.error('Error:', error);
        }
    };

    const handleJoinSession = async () => {
        try {
            const response = await fetch('http://localhost:3002/login/participant', {
                method: 'POST',
                headers: {
                    'Content-Type': 'application/json'
                },
                body: JSON.stringify(ob_json)
            });

            if (!response.ok) {
                console.log(response.body)
                throw new Error('Network response was not ok');
            }
            const data = await response.json();
            dispatch(setUserInfo({ username: username, userType: "user", isLoggedIn: true, sessionId: sessionId }));
            console.log(data); // Handle the response data
        } catch (error) {
            console.error('Error:', error);
        }
    };

    return (
        <div className='container min-h-screen min-w-full justify-center content-center flex'>
            <Card className="w-[350px] h-fit self-center">
                <CardHeader>
                    <CardTitle>Calendar Sync</CardTitle>
                    <CardDescription>Multi-client real-time calendar application. Create a new calendar or join exisiting one</CardDescription>
                </CardHeader>
                <CardContent>
                    <form>
                        <div className="grid w-full items-center gap-4">
                            <div className="flex flex-col space-y-1.5">
                                <Label htmlFor="uuid">UUID for existing session</Label>
                                <Input id="uuid" placeholder="XXXX-XXXX-XXXX" />
                            </div>
                        </div>
                    </form>
                </CardContent>
                <CardFooter className="
                                    flex 
                                    justify-between
                                    ">
                    <Button className='hover:bg-green-700' onClick={handleCreateSession}>
                        <SquarePlus className='mr-2 ' />Create
                    </Button>
                    <Button className='hover:bg-teal-300' variant="outline">
                        <Merge className='mr-2' />
                        Join
                    </Button>
                </CardFooter>
            </Card>
        </div>
    );
};
\end{lstlisting}
\begin{lstlisting}[language=Java, caption={Socket Provider Component}, label={lst:java}]
'use client'
import { createContext, useContext, useEffect, useState } from "react";
import { io as ClientIO } from 'socket.io-client'

export type SocketContextType = {
    socket: any | null,
    isConnected: boolean
}

const defaultSocket = {
    socket: null,
    isConnected: false
};

export const SocketContext = createContext<SocketContextType>(defaultSocket);

export const useSocket = () => {
    return useContext(SocketContext);
}

export const SocketProvider= ({ children }: { children: React.ReactNode }) => {
    const [socket, setSocket] = useState<any | null>(null);
    const [isConnected, setIsConnected] = useState(false);

    useEffect(() => {
        const socketInstance = new (ClientIO as any)(`http://localhost:5050`);
        console.log("Creating websocket...")
        
        // Add an event listener to handle connections
        socketInstance.on('connect', () => {
            console.log("Websocket connected    !");
            setIsConnected(true);
        });
        
        socketInstance.on('disconnect', (reason: any) => {
            console.warn(`Websocket disconnected because of ${reason}`);
            setIsConnected(false);
        });

        setSocket(socketInstance);
        
        //Clean up function that gets called whenever this component unmounts
        return () => {
            console.log("Disconnecting websocket...");
            // Close the connection when this component unmounts
            socketInstance.disconnect();
        };

     }, []);
     
    return (
        <SocketContext.Provider value={{socket: socket, isConnected: isConnected}}>
            {children}
        </SocketContext.Provider>
    )
}
\end{lstlisting}
\begin{lstlisting}[language=Java, caption={Express Backend server}, label={lst:java}]
const express = require('express');
const http = require('http');
const socketIo = require('socket.io');
const cors = require('cors');
const bodyParser = require('body-parser');

const app = express();
app.use(cors());
const server = http.createServer(app);
const port = 5050
const io = socketIo(server, {
    cors: {
        origin: "*",
        methods: ["GET", "POST"]
    }
});
app.use(bodyParser.json());

const activeRooms = {};

app.get("/session", (request, response) => {
    const { username } = request.body;

    const user = users.find(user => user.username === username);
    if (user) {
        const session = userSessions.get(user.username);
        if (session) {
            response.json({ session });
        } else {
            response.status(404).json({ error: 'Session not found' });
        }
    }
    else {
        response.status(404).json({ error: 'User not found' });
    }
});

app.post("/session", (request, response) => {
    const data = request.body;
    console.log(data);
    const user = users.find(user => user.username === data.username);

    if (user) {
        const sessionId = Math.random().toString(36).substring(7);
        const participants = data.participants;

        if (participants) {
            const temp_users: L_User[] = [];
            for (const p of participants) {
                temp_users.push({ username: p.username, password: p.password })
            }
            const session: L_Session = { id: sessionId, participants: temp_users };
            console.log(`Session create: ${sessionId}, Users: ${temp_users.toString()}`)
            sessionParticipants.push(session);
            userSessions.set(user.username, sessionId);
        }
        response.json({ sessionId });
    } else {
        response.status(404).json({ error: 'User not found' });
    }
});

app.get('/chatroom/:sessionId', (req, res) => {
    const sessionId = req.params.sessionId;
    if (!sessionId) {
       return res.status(400).send('Missing sessionId parameter');
    }
  
    // Create a new namespace for the chatroom with the sessionId
    const chatroom = io.of(`/chatroom/${sessionId}`);
  
    chatroom.on('connection', (socket) => {
      console.log(`Socket connected to chatroom: ${sessionId}`);
  
      // Handle socket events specific to this chatroom
      socket.on('message', (data) => {
        console.log(`Message received in ${sessionId}: ${data}`);
        // Broadcast the message to all connected sockets in the chatroom
        chatroom.emit('message', data);
      });
  
      // Handle socket disconnection
      socket.on('disconnect', () => {
        console.log(`Socket disconnected from chatroom: ${sessionId}`);
      });
    });
  
    // Send a success response to the UI
    res.send('Chatroom created');
  });


io.on('connection', (socket) => {
    let roomId;

    socket.on('createRoom', (payload) => {
        console.log(`MSG createRoom => Payload recieved: ${JSON.stringify(payload)}`);
    });

    socket.on('joinRoom', (payload) => {

        const room = activeRooms[payload];
        if(!room){
            activeRooms[payload] = {events:[]};
            console.log(`New room created: ${payload}`);
            activeRooms[payload].userCount = 1;
        }
        console.log(`All rooms: ${JSON.stringify(activeRooms)}`);
        if (room) {
            socket.join(payload);
            // Send chat history to the new user
            const events = activeRooms[payload].events;
            socket.emit('allEvents', events);
            activeRooms[payload].userCount++;
        }
        console.log(`MSG joinRoom => Payload recieved: ${JSON.stringify(payload)}`);
    });

    // Handler for sending a message
    socket.on('updateEvents', (payload) => {
        const { sessionId, newEvents} = payload;
        const room = activeRooms[sessionId];
        if (room) {
            console.log(`Updating events for room: ${sessionId}`);
            room.events= newEvents;
            io.to(sessionId).emit('updateEvents', newEvents);
        }
        console.log(`All rooms: ${JSON.stringify(activeRooms)}`);
        console.log(`MSG updateEvents => Payload received: ${JSON.stringify(payload)}`);
    });

    // Handler for disconnecting
    socket.on('disconnect', () => {
        console.log(`Disconnecting...`);
    });
});

server.listen(port, () => {
    console.log(`Server is running on port ${port}`);
});

% \end{lstlisting}
% \begin{lstlisting}[language=Java, caption={Login Page}, label={lst:java}]
\end{lstlisting}
