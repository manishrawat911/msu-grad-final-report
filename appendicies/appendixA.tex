% !TEX root = ../main.tex

%----------------------------------------
%   APPENDIX TITLE
%----------------------------------------
\chapter{TEASync Components and Backend Server}
% {TEASync Components and Backend Server}
\label{appendix:A}
% \section{TEASync Components and Backend Server}
%----------------------------------------
%   APPENDIX CONTENT
%----------------------------------------
\vspace{4pt}
\begin{lstlisting}[language=Haskell, caption=Defining a Event component in TEASync Calendar Application, label=lst:java]
drawMeeting : (Day,Start) -> Meeting -> Bool -> Shape (TEASync.Msg LocalMsg GlobalMsg GlobalModel)
drawMeeting dayStart (Meet location duration attendance) dragging =
  let
    (x,y) = time2pos dayStart
    (_,start) = dayStart
    hour = start // 60
    minutes = start - (60 * hour)
    displayTime = String.fromInt (9+hour) 
                ++ ":" 
                ++ ("0"++String.fromInt minutes 
                     |> String.right 2          )
    
    height = duration2height duration 
    locShape =[ roundedRect 12 4 2
                  |> filled (rgb 0 0 255)
              , ( case location of
                   Teams -> "Teams"
                   Lab -> "Lab"
                   PGCLL -> "PGCLL"
                ) |> text |> size 2
                  |> filled white
                  |> move (-5,-0.5)
              ] |> group
                |> move (-7, -9 + 0.5 * height)
                |> notifyTap (TapToToggleLocation dayStart |> GlobalMsg)
    attendeeShapes = attendance
      |> Dict.toList
      |> List.indexedMap 
            ( \ idx (person,isAttending) -> 
                let
                  (clr, check) = if isAttending then
                                   (green,"✓")
                                 else
                                   (red,"��")
                in [ roundedRect 12 4 2
                                  |> filled clr
                              , ( check ++ person) |> text |> size 2
                                  |> filled black
                                  |> move (-5,-0.5)
                              ] |> group
                                |> move (7, -4 - 5 * toFloat idx + 0.5 * height)
                                |> notifyTap (TapToToggleAttendance dayStart person |> GlobalMsg)
           )
  in
    [ roundedRect (dayWidth-1) height 1
          |> filled orange
          |> (if dragging then addOutline (solid 1) red else identity)
    , displayTime |> text |> size 4 |> filled black
                  |> move (-13, -5.5 + 0.5 * height)
    -- topGrip
    , grip |> move (0,0.5 * height)
           |> notifyMouseDownAt (MouseDownToAdjustStartAt dayStart >> LocalMsg)
    , grip |> scaleY (-1)
           |> move (0,-0.5 * height)
           |> notifyMouseDownAt (MouseDownToAdjustStopAt dayStart >> LocalMsg)
    , locShape 
    ] ++ attendeeShapes
      |> group
      |> move (x,y - 0.5 * height)
\end{lstlisting}

\begin{lstlisting}[language=Haskell, caption=TEASync Backend server (WSServer.hs), label=lst:java]
    
{-# LANGUAGE DeriveGeneric, DeriveAnyClass #-}

module Application.TEASync.Types.WSServer where

import IHP.Prelude
import IHP.ModelSupport
import Generated.Types

import Control.Concurrent.STM
import Data.Aeson
import GHC.Generics

type ClientID = Int
type ServerID = Id TeasyncServer
type WSClientQueue = TQueue WSClientCmd

-- actions that can be queued up to be sent to Elm clients
data WSClientCmd 
    = SendConnected
    | SendModel { model :: Value }
    | SendMsg { id :: Text, message :: Value }
    | SendModelRequest
    | CloseWS

-- message from Elm client
data WSClientMsg =
    WSClientMsg
        {
            cmd :: Text
        ,   id  :: Text
        ,   msg :: Value
        }
    deriving (Generic, Show, Eq, FromJSON)

-- actions that can be received from Elm clients
data WSCmd
    = ReceiveMsg { message :: Text }
    | ReceiveModel { model :: Text }
    deriving (Generic, Show, FromJSON)

data ReceiveConnected
    = ReceiveClientConnected { v :: Text, serverUUID :: UUID }
    | ReceiveHeadlessConnected { key :: Text, serverUUID :: UUID }
    deriving (Generic, Show)

instance FromJSON ReceiveConnected where
  parseJSON = withObject "ReceiveConnected" $ \o -> do
    v <- o .:? "v"
    serverUUID <- o .: "serverUUID"
    key <- o .:? "key"
    case v of
      Just v' -> return $ ReceiveClientConnected v' serverUUID
      Nothing -> case key of
        Just key' -> return $ ReceiveHeadlessConnected key' serverUUID
        Nothing -> fail "Invalid JSON"

-- create a JSON instance for WSClientCmd
instance ToJSON WSClientCmd where
    toJSON SendConnected = object [ "cmd" .= ("c" :: Text) ]
    toJSON CloseWS = object [ "cmd" .= ("x" :: Text) ]
    toJSON (SendModel model) = object [ "cmd" .= ("m" :: Text), "model" .= model ]
    toJSON (SendMsg id message) = 
        object 
            [ "cmd" .= ("s" :: Text)
            , "msg" .= message 
            , "id"  .= id
            ]
    toJSON SendModelRequest = object [ "cmd" .= ("r" :: Text) ]


-- state of a single WebSocket connection
data TSWSServerController
    = WaitingToConnect
    | ClientConnected 
        { 
            serverId :: ServerID
        ,   clientId :: ClientID
        ,   isHeadless :: Bool
        }
    deriving (Eq, Show, Data)

-- state of the entire WebSocket server
data TSWSServerState
    = WSServerState
        {
            nextId :: ClientID
        ,   clientMap :: HashMap ServerID (HashSet ClientID)
        ,   clientChannels :: HashMap ClientID WSClientQueue
        ,   headlessClientIDs :: HashMap ServerID ClientID
        ,   serverModels :: HashMap ServerID Data.Aeson.Value
        ,   unstoredModels :: HashSet ServerID
        ,   unstoredModelThreadStarted :: Bool
        }

data ModelStorerMsg = NewModelStored

\end{lstlisting}